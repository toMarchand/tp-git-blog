\documentclass[a4paper]{article}
\usepackage[utf8]{inputenc}
\usepackage[T1]{fontenc}
\usepackage{graphics}
\usepackage[ttscale=.875]{libertine}
\usepackage{tango}
%\usepackage[colorlinks=true]{hyperref}

\hypersetup{urlcolor=DarkChocolate}

\title{TP Git}
%\author{\scalebox{0.35}{ \includegraphics{logo_azae}}}
\author{Thomas Clavier}
\date{}

\begin{document}

\maketitle

\section{configuration}

À l'aide de la commande <<git config>> configurez votre nom (user.name) votre email (user.email) et le serveur mandataire (http.proxy) dans git. L'URL du serveur mandataire de l'université est http://cache.univ-lille1.fr:3128

Vérifier à l'aide de la commande <<git config --list>>

\section{L'espace de travail}

À partir de là nous allons construire un blog pour le groupe de TP.

\subsection{Initialisation}
Créez un répertoire de travail et initialisez le comme un dépôt git local.

\subsection{Historique de version}
Dans ce blog, tous les articles sont rédigés en utilisant le format Markdown (\url{http://daringfireball.net/projects/markdown/syntax}).

\subsubsection{Premier article}
Dans l'espace de travail que vous avez initialisé comme dépôt local, créez un sous répertoire <<content>>.

Avec votre voisin choisissez un sujet, puis créez le même article <<content/votre-sujet.md>> chacun dans votre dépôt local avec le contenu suivant : 

\begin{verbatim}
+++
date = "2014-01-01T12:00:00+01:00"
draft = true
title = "Mon sujet"
+++

# Un titre
\end{verbatim}

Attention, la date de publication est au format iso 8601.

Vous pouvez par exemple faire un article avec les réponses à ce TP.

À l'aide de <<git add>>, <<git commit>> et <<git log>>, l'ajouter dans l'index comme fichier à suivre, puis validez le changement. 
La commande <<git log --oneline --decorate --graph --all>> devrait vous donnez :
\begin{verbatim}
* a0712c7 (HEAD -> master) Add my first article
\end{verbatim}

Observez l'historique des changements, quelles sont les informations disponibles ? À quoi correspondent les champs ?

\subsubsection{Second changement}
Ajoutez quelques informations dans votre article puis validez un nouveau commit.

Par exemple : 
\begin{verbatim}
+++
date = "2014-01-01T12:00:00+01:00"
draft = true
title = "Mes réponses au TP Git"
+++

# TP Git

## Configuration

    git config --global user.name "Pierre Dupond"
    git config --global user.email "pierre.dupond@univ-lille1.fr"
    git config --global http.proxy http://cache.univ-lille1.fr:3128

\end{verbatim}

La commande <<git log --oneline --decorate --graph --all>> devrait vous donnez :
\begin{verbatim}
* b7f0aaa (HEAD -> master) Add some content in my article
* a0712c7 Add my first article
\end{verbatim}

Comment obtenir le diff entre les 2 derniers commits ?

\subsubsection{Suppression}
\begin{itemize}
\item Créez un article <<content/lego.md>> puis validez un nouveau commit. 
\item Supprimez l'article <<content/lego.md>> puis validez un nouveau commit. 
\end{itemize}

La commande <<git log --oneline --decorate --graph --all>> devrait vous donnez :
\begin{verbatim}
* 8c18094 (HEAD -> master) Remove article about Lego
* c33514e Add article about Lego
* b7f0aaa Add some content in my article
* a0712c7 Add my first article
\end{verbatim}

Que voit-on dans l'historique ?
Comment, à l'aide de la commande <<git checkout>> peut-on retrouver le fichier en question ?

La commande <<git log --oneline --decorate --graph --all>> devrait vous donnez :
\begin{verbatim}
* 8c18094 (master) Remove article about Lego
* c33514e (HEAD) Add article about Lego
* b7f0aaa Add some content in my article
* a0712c7 Add my first article
\end{verbatim}

\subsubsection{Retour en arrière}

Revenir à la situation tel que la commande <<git log --oneline --decorate --graph --all>> vous donne :
\begin{verbatim}
* 8c18094 (HEAD -> master) Remove article about Lego
* c33514e Add article about Lego
* b7f0aaa Add some content in my article
* a0712c7 Add my first article
\end{verbatim}

Ajoutez quelques informations dans votre article sans valider un nouveau commit.

Comment faire pour annuler le travail et revenir à la dernier version <<commité>> du fichier ?

La commande <<git log --oneline --decorate --graph --all>> devrait vous donnez :
\begin{verbatim}
* 8c18094 (HEAD -> master) Remove article about Lego
* c33514e Add article about Lego
* b7f0aaa Add some content in my article
* a0712c7 Add my first article
\end{verbatim}

\begin{itemize}
  \item Créez un article <<content/satoshi-tajiri.md>> puis validez un nouveau commit.
\end{itemize}

Comment faire pour annuler ce dernier commit et revenir avec une copie de travail sans ce dernier fichier en HEAD ?

La commande <<git log --oneline --decorate --graph --all>> devrait vous donnez :
\begin{verbatim}
* 655dc4c (HEAD -> master) Cancel "Add personal details in my article"
* 8d292f3 Add personal details in my article
* 8c18094 Remove article about Lego
* c33514e Add article about Lego
* b7f0aaa Add some content in my article
* a0712c7 Add my first article
\end{verbatim}

\section{Partager}

\subsection{Via un dépôt partagé}

Indiquons à Git que nous souhaitons échanger des données avec le dépôt distant suivant : \begin{verbatim}http://iut.azae.net/git/tp-git.git\end{verbatim}
Nous appelons ce dépôt <<origin>>.

Puis envoyez l'ensemble des modifications de la branche courante (master) vers le dépôt <<origin>>

Que se passe-t-il ?

Visitez \url{http://iut.azae.net/} pour admirer votre travail.

\subsection{Via un second dépôt}

Aller sur un serveur publique comme \url{https://github.com/}, \url{https://bitbucket.org/} ou \url{https://gitlab.com}, une fois connecté créez un nouveau repository : tp-git-blog. Puis ajouter ce nouveau remote sous le nom <<kura>>. Faire un push vers ce dépôt et observer dans l'interface web le résultat.


\end{document}

